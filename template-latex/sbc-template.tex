\documentclass[12pt]{article}

\usepackage{sbc-template}

\usepackage{graphicx,url}

%\usepackage[brazil]{babel}   
\usepackage[latin1]{inputenc}  

     
\sloppy

\title{\\ A survey for the essential Data Mining technique: Clustering}

\author{Isabel B. Amaro\inst{1}, Mirella M. Moro\inst{2}, Clodoveu Davis\inst{3}}


\address{Department of Computer Science -- Federal University of Minas Gerais
  (UFMG)\\
  Belo Horizonte, Brazil
  \email{\{isabel.amaro,mirella,clodoveu\}@dcc.ufmg.br}
}

\begin{document} 

\maketitle

\begin{abstract}
Clustering is a Data Mining technique capable of group data with some non trivial similarity. Because of the advance of technology and data generation, clustering algorithms had to be developed and improved to process and extract useful information from the current large amount of data. Therefore, many areas that use Data Mining are constantly publishing articles with new clustering techniques and algorithms. This survey will study some state-of-the-art clustering algorithms, summarize, organize and present the temporal analysis in order to support the studies of its readers.
\end{abstract}

\section{Introduction}
Clustering is a Data Mining technique capable of group data with some non trivial similarity. Because of the advance of technology and data generation, clustering algorithms had to be developed and improved to process and extract useful information from the current large amount of data. Therefore, many areas that use Data Mining are constantly publishing articles with new clustering techniques and algorithms. This survey will study some state-of-the-art clustering algorithms, summarize, organize and present the temporal analysis in order to support the studies of its readers.

Dados vem sendo gerados de forma exponencialmente nos últimos anos devido ao
Com o avanço da tecnologia, 
Um dos pontos de interesse principais do avanço da tecnologia é o oferecimento de serviços personalizados baseados nas informações adquiridas, podendo essas serem específicas de um usuário, de um texto ou de um banco de dados, por exemplo. Devido a isso, 

\section{Different clustering case of use}

\section{Clustering}

\section{Data Mining}

\section{Algorithms}

\subsection{K-Means}

\subsection{Hierarchical clustering}

\subsection{DBSCAN}

\section{Temporal analysis}

\section{Conclusion}

%\section{References}

%Bibliographic references must be unambiguous and uniform.  We recommend giving
%the author names references in brackets, e.g. \cite{knuth:84},
%\cite{boulic:91}, and \cite{smith:99}.

%The references must be listed using 12 point font size, with 6 points of space
%before each reference. The first line of each reference should not be
%indented, while the subsequent should be indented by 0.5 cm.

\bibliographystyle{sbc}
%\bibliography{sbc-template}

\end{document}
