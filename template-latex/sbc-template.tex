\documentclass[12pt]{article}

\usepackage{sbc-template}

\usepackage{graphicx,url}

%\usepackage[brazil]{babel}   
\usepackage[latin1]{inputenc}  

     
\sloppy

\title{\\ Algorithms, the Classics never Die: A survey to analyse the state of the art of clustering}

\author{Isabel B. Amaro\inst{1}, Mirella M. Moro\inst{2}, Clodoveu Davis\inst{3}}


\address{Department of Computer Science -- Federal University of Minas Gerais
  (UFMG)\\
  Belo Horizonte, Brazil
  \email{\{isabel.amaro,mirella,clodoveu\}@dcc.ufmg.br}
}

\begin{document} 

\maketitle

\begin{abstract}
In the last years, computing and its algorithms have evolved more and more rapidly, and nowadays, Data Mining has taken up a lot of space in Computer Science due to the need to handle the large amount of information generated by the advancement of technology. This article will study the state of the art of some algorithms used for Clustering, a technique widely used in Data Mining, and synthesize the obtained results in order to support the studies of its readers.
\end{abstract}

\section{Introduction}

\section{Clustering}

\section{Data Mining}

\subsection{Text Mining}

\section{Algorithms}

\subsection{K-Means}

\subsection{Prim}

\section{Conclusion}

\section{References}

Bibliographic references must be unambiguous and uniform.  We recommend giving
the author names references in brackets, e.g. \cite{knuth:84},
\cite{boulic:91}, and \cite{smith:99}.

The references must be listed using 12 point font size, with 6 points of space
before each reference. The first line of each reference should not be
indented, while the subsequent should be indented by 0.5 cm.

\bibliographystyle{sbc}
\bibliography{sbc-template}

\end{document}
